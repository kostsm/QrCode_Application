\documentclass[10pt]{beamer}
\usepackage[T1,T2A]{fontenc}
\usepackage[utf8]{inputenc}
\usepackage{hyperref}
\hypersetup{unicode=true}
\usepackage{fontawesome}
\usepackage{graphicx}
\usepackage[english,russian]{babel}

\usepackage[T1]{fontenc}
\usepackage{fontawesome}
\usepackage{PTSans} 
\mode<presentation>
{
  \usetheme[progressbar=foot,numbering=fraction,background=light]{metropolis} 
  \usecolortheme{default}
  \usefonttheme{default}
  \setbeamertemplate{navigation symbols}{}
  \setbeamertemplate{caption}[numbered]
} 

\let\textttorig\texttt
\renewcommand<>{\texttt}[1]{%
  \only#2{\textttorig{#1}}%
}

\usepackage{minted}

\usepackage{xcolor}
\definecolor{codecolor}{HTML}{FFC300}

\usepackage{tcolorbox}
\tcbuselibrary{most,listingsutf8,minted}

\tcbset{tcbox width=auto,left=1mm,top=1mm,bottom=1mm,
right=1mm,boxsep=1mm,middle=1pt}

\newtcblisting{myr}[1]{colback=codecolor!5,colframe=codecolor!80!black,listing only, 
minted options={numbers=left, style=tcblatex,fontsize=\tiny,breaklines,autogobble,linenos,numbersep=3mm},
left=5mm,enhanced,
title=#1, fonttitle=\bfseries,
listing engine=minted,minted language=r}

\definecolor{mygreen}{HTML}{37980D}
\definecolor{myblue}{HTML}{0D089F}
\definecolor{myred}{HTML}{98290D}

\usepackage{listings}

\lstdefinelanguage{XML}
{
  morestring=[b]",
  morecomment=[s]{<!--}{-->},
  morestring=[s]{>}{<},
  morekeywords={ref,xmlns,version,type,canonicalRef,metr,real,target}
}

\lstdefinestyle{myxml}{
language=XML,
showspaces=false,
showtabs=false,
basicstyle=\ttfamily,
columns=fullflexible,
breaklines=true,
showstringspaces=false,
breakatwhitespace=true,
escapeinside={(*@}{@*)},
basicstyle=\color{mygreen}\ttfamily,
stringstyle=\color{myred},
commentstyle=\color{myblue}\upshape,
keywordstyle=\color{myblue}\bfseries,
}


% ------------------------------------------------------------------------------
% The Document
% ------------------------------------------------------------------------------
\title{Разработка приложения для генерации и сканирования QR-кодов}
\subtitle{Отчет о проектной работе по курсу <<Разработка приложений для мобильных операционных систем>>}
\author{Е. Ф. Волкова, К. А. Смирнов, Е. Д. Топчий}
\date{17 января 2022}

\begin{document}

\maketitle

\begin{frame}{Введение}
Большинство современных смартфонов может считывать коды с помощью наведения камеры, устройство распознает изображение и выдает результат. Специалисты предрекают большое будущее этой системе контроля и предоставления информации, поскольку она сегодня одобрена на государственном уровне и используется в борьбе с пандемией вируса. В скором времени сканеры станут привычными и повсеместными средствами способными предоставить всю необходимую информацию. В данном проекте речь пойдет о создании функционала для работы с QR-кодами.

\end{frame}

\begin{frame}
  \frametitle{Цель и задачи проекта}
    \begin{block}{Цель проекта:}
        разработать приложение для сканирования и генерации QR-кодов  на языке Kotlin в среде разработке Android Studio.
  \end{block}
  \begin{block}{Задачи проекта:}
  \begin{itemize}
    \item изучить вариации реализации приложения и на их примере разработать требования к собственному приложению.
    \item разработать графический интерфейс пользователя.
    \item реализовать приложение с использованием разработанных модулей и необходимых библиотек на языке программирования Kotlin.
    \item получить навыки по составлению документации, описывающей работу программы;
  \end{itemize}
  \end{block}
\end{frame}

\begin{frame}{План разработки}
    \begin{enumerate}
         \item Разработка модуля, отвечающего за генерацию QR-кода.
         \item Разработка модуля, отвечающего за сканирование QR-кода.
         \item Разработка модуля, реализующего вывод информации о QR-коде.
         \item Разработка модуля, отвечающего за разрешение использования камеры.
         \item Разработка модуля, отвечающего за диалоговые окна.
         \item Разработка модуля пользовательского интерфейса.
    \end{enumerate}
\end{frame}

\begin{frame}{MainActivity}
        \begin{columns}[T,onlytextwidth]
                \begin{column}{0.44\textwidth}
                        \begin{itemize}
                                \item Отображение стартового окна с настройками и кнопками запуска сканирования и генерации соотвественно.
                                \item Модуль, отвечающий за генерацию QR-кода (возможен ввод как текста, так и ссылки) и за вывод его пользователю.
                                \item Реализация анимации градиента для фона.
                        \end{itemize}
                \end{column}
                \begin{column}{0.3\textwidth}
                        \includegraphics[width=\textwidth]{2.jpg}
                \end{column}
        \end{columns}
\end{frame}

\begin{frame}{ScannerActivity}
    Запуск при нажатии кнопки считывания QR-кода в стартовом окне приложения, отображание превью камеры и определение QR-кода на нем.\\ 
    \\
        При сканировании QR-кода пользователем появится диалоговое окно,
        на котором будет отображена ссылка или текст, зашифрованные в данном QR-коде,
        а так будет предложено 3 действия:\\
    \begin{itemize}
        \item Перейти по ссылке(в случае, если в QR-коде заложен обычный текст,
        перейдет по ссылке в Google поиск с запросом в виде данного текста)\\
        \item Скопировать ссылку или текст - копирует либо ссылку, либо текст.\\
        \item Назад - возвращает пользователя в главное меню приложения
    \end{itemize}
\end{frame}

\begin{frame}{ScannerActivity}
\begin{figure}[h]
\begin{center}
\begin{minipage}[h]{0.28\linewidth}
\includegraphics[width=1\linewidth]{4.jpg}
\caption{Окно для сканирования} %% подпись к рисунку
\label{ris:experimoriginal} %% метка рисунка для ссылки на него
\end{minipage}
\hfill
\begin{minipage}[h]{0.28\linewidth}
\includegraphics[width=1\linewidth]{5.jpg}
\caption{Диалоговое окно}
\label{ris:experimcoded}
\end{minipage}
\end{center}
\end{figure}
\end{frame}

\begin{frame}{Другие модули}
    \begin{columns}[T,onlytextwidth]
                \begin{column}{0.44\textwidth}
                        \begin{itemize}
                                \item Также были задействованы другие модули для разрешения доступа на использование камеры и подключения библиотек.
                                \item При первом запуске приложения будет появляться диалоговое окно с разрешением доступа к камере.
                                \item Возможны три действия: разрешение, отклонение или запрет навсегда.
                        \end{itemize}
                \end{column}
                \begin{column}{0.3\textwidth}
                        \includegraphics[width=\textwidth]{3.jpg}
                \end{column}
        \end{columns}
\end{frame}

\begin{frame}{Реализация приложения}
        \begin{columns}[T,onlytextwidth]
                \begin{column}{0.44\textwidth}
                        \begin{itemize}
                                \item Так, нами был разработан функционал для генерации и сканирования QR-кодов - приложение, с помощью которого можно считывать информацию одним наведением цифровой видеокамеры. 
                                \item Для этого был использован язык Kotlin и среда разработки Android studio.
                        \end{itemize}
                \end{column}
                \begin{column}{0.3\textwidth}
                        \includegraphics[width=\textwidth]{1.jpg}
                \end{column}
        \end{columns}
\end{frame}

\begin{frame}
    \frametitle{Заключение}
    Реализация программы способствовало закреплению теоретического материала на практике. Также возможны изменения и добавления некоторых функциональных частей, которые можно реализовать в дальнейшем.
\end{frame}

\begin{frame}[standout]
    \begin{figure}[h]
   \centering
   \includegraphics[width=1\textwidth]{Снимок.JPG}
   \caption{\label{fig:trans-vs-reactive}}
\end{figure}
\end{frame}

\end{document}
