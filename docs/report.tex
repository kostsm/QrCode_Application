%%% Для сборки выполнить 2 раза команду: pdflatex <имя файла>

\documentclass[a4paper,12pt]{article}

\usepackage{ucs}
\usepackage[utf8x]{inputenc}
\usepackage[russian]{babel}
%\usepackage{cmlgc}
\usepackage{graphicx}
\usepackage{hyperref}
\usepackage{listings}
\usepackage{xcolor}
%\usepackage{courier}

\makeatletter
\renewcommand\@biblabel[1]{#1.}
\makeatother

\newcommand{\myrule}[1]{\rule{#1}{0.4pt}}
\newcommand{\sign}[2][~]{{\small\myrule{#2}\\[-0.7em]\makebox[#2]{\it #1}}}

% Поля
\usepackage[top=20mm, left=30mm, right=10mm, bottom=20mm, nohead]{geometry}
\usepackage{indentfirst}

% Межстрочный интервал
\renewcommand{\baselinestretch}{1.50}


\begin{document}

%%%%%%%%%%%%%%%%%%%%%%%%%%%%%%%
%%%                         %%%
%%% Начало титульного листа %%%

\thispagestyle{empty}
\begin{center}


\renewcommand{\baselinestretch}{1}
{\large
{\sc Петрозаводский государственный университет\\
Институт математики и информационных технологий\\
	Кафедра Информатики и Математического Обеспечения
}
}

\end{center}


\begin{center}
%%%%%%%%%%%%%%%%%%%%%%%%%
%
% Раскомментируйте (уберите знак процента в начале строки)
% для одной из строк типа направления  - бакалавриат/
% магистратура и для одной из
% строк Вашего направление подготовки
%
% Направление подготовки бакалавриата \\
% 01.03.02 Прикладная математика и информатика \\
% 09.03.02 - Информационные системы и технологии \\
09.03.04 - Программная инженерия \\
% Направление подготовки магистратуры \\
% 01.04.02 - Прикладная математика и информатика \\
% 09.04.02 - Информационные системы и  технологии \\
%
% 
%%%%%%%%%%%%%%%%%%%%%%%%%
	% \textcolor{red}{<Ваши тип и направление подготовки>} 
\end{center}

\vfill

\begin{center}
{\normalsize Отчет о проектной работе по курсу <<Разработка приложений для мобильных операционных систем>>} \\

\medskip

%%% Название работы %%%
	{\Large \sc Разработка приложения для генерации и сканирования QR-кодов} \\
	% (промежуточный)
\end{center}

\medskip

\begin{flushright}
\parbox{11cm}{%
\renewcommand{\baselinestretch}{1.2}
\normalsize
	Выполнили:\\
%%% ФИО студента %%%
студенты 2 курса группы 22207
\begin{flushright}
	Е. Ф. Волкова \sign[подпись]{4cm}\\
	К. А. Смирнов \sign[подпись]{4cm}\\
	Е. Д. Топчий \sign[подпись]{4cm}
\end{flushright}

Руководитель:\\
А. В. Бородин, старший преподаватель \\
% \begin{flushright}
% \sign[подпись]{4cm}
% \end{flushright}

}
\end{flushright}

\vfill

\begin{center}
\large
    Петрозаводск --- 2022
\end{center}

%%% Конец титульного листа  %%%
%%%                         %%%
%%%%%%%%%%%%%%%%%%%%%%%%%%%%%%%

%%%%%%%%%%%%%%%%%%%%%%%%%%%%%%%%
%%%                          %%%
%%% Содержание               %%%

\newpage

\hypersetup{hidelinks}
\tableofcontents

\newpage
\section*{Введение}
\addcontentsline{toc}{section}{Введение}


Цель проекта: разработать приложение для сканирования и генерации QR-кодов  на языке Kotlin в среде разработке Android Studio. \\

Задачи проекта: 
\begin{enumerate}
    \item Изучить вариации реализации приложения и на их примере разработать требования к собственному приложению.
    \item Разработать графический интерфейс пользователя.
    \item Реализовать приложение с использованием разработанных модулей и необходимых библиотек на языке программирования Kotlin.
    \item Получить навыки по составлению документации, описывающей работу программы. 
\end{enumerate}

Сканеры для считывания кодов стали необходимостью, поскольку их все чаще размещают на важных объектах, товарах и информационных объявлениях. Сканер дает возможность:
\begin{itemize}
    \item перейти на адрес, зашифрованный в изображении и получить информацию;
    \item прочитать справочную информацию, если изображение ее содержит;
    \item для заполнения бланков и оплаты коммунальных услуг;
\end{itemize}

Большинство современных смартфонов может считывать коды с помощью наведения камеры, устройство распознает изображение и выдает результат. Специалисты предрекают большое будущее этой системе контроля и предоставления информации, поскольку она сегодня одобрена на государственном уровне и используется в борьбе с пандемией вируса. В скором времени сканеры станут привычными и повсеместными средствами способными предоставить всю необходимую информацию. В данном проекте речь пойдет о создании функционала для работы с QR-кодами.


%%%                          %%%
%%%%%%%%%%%%%%%%%%%%%%%%%%%%%%%%

\newpage

%%%%%%%%%%%%%%%%%%%%%%%%%%%%%%%%
%%%                          %%%
%%% Требования к приложению  %%%

\section{Требования к приложению}
\begin{itemize}
    \item Возможность кодирования любой информации (текст, ссылка и т.п.).
    \item Вывод сгенерированного QR-кода.
    \item Сканирование необходимого QR-кода с помощью камеры и вывод информации о нем.
    \item Приятный интерфейс.
    
\end{itemize}

%%%                          %%%
%%%%%%%%%%%%%%%%%%%%%%%%%%%%%%%%


%%%%%%%%%%%%%%%%%%%%%%%%%%%%%%%%%
%%%                           %%%
%%% Проектирование приложения %%%
\section{Проектирование приложения}
Программа будет состоять из следующих основных функциональных частей:
\begin{itemize}
    \item Модуль, отвечающий за генерацию QR-кода.
    \item Модуль, отвечающий за сканирование QR-кода.
    \item Модуль, реализующий вывод информации о QR-коде.
    \item Модуль, отвечающий за доступ к камер.
    \item Модуль, отвечающий за разрешение камеры.
    \item Модуль, отвечающий за дизайн (градиент, т.д.).
    \item Модуль, отвечающий за диалоговые окна (доступ к камере, взаимодействие со считанной информации).
\end{itemize}

%%%                          %%%
%%%%%%%%%%%%%%%%%%%%%%%%%%%%%%%%

\newpage

%%%%%%%%%%%%%%%%%%%%%%%%%%%%%%%%%
%%%                           %%%
%%% Реализация приложения     %%%
\section{Релизация приложения}
Для реализации приложения был использован язык программирования Kotlin в среде разботке Android Studio.
    \begin{enumerate}
        \item MainActivity --- отображение стартового окна с настройками и кнопками запуска сканирования и генерации соотвественно, функция генерации QR-кода.
        \item ScannerActivity ---  запуск при нажатии кнопки считывания QR-кода в стартовом окне приложения, отображание превью камеры и определение QR-кода на нем.\\
        При сканировании QR-кода пользователем появится диалоговое окно,
        на котором будет отображена ссылка или текст, зашифрованные в данном QR-коде,
        а так будет предложено 3 действия:
    \begin{itemize}
        \item Перейти по ссылке(в случае, если в QR-коде заложен обычный текст,
        перейдет по ссылке в Google поиск с запросом в виде данного текста)
        \item Скопировать ссылку или текст - копирует либо ссылку, либо текст.
        \item Назад - возвращает пользователя в главное меню приложения
    \end{itemize}
        \item activity\_main.xml --- интерфейс приложения.
        \item Также были задействованы другие модули для разрешения доступа на использование камеры и подключения библиотек.При первом запуске приложения будет появляться диалоговое окно с разрешением доступа к камере. Возможны три действия: разрешение, отклонение или запрет навсегда.
    \end{enumerate}


%%%                          %%%
%%%%%%%%%%%%%%%%%%%%%%%%%%%%%%%%

\begin{figure}[H]
\begin{minipage}[h]{0.32\linewidth}
\center{\includegraphics[width=0.9\linewidth]{1.jpg}}
\end{minipage}
\hfill
\begin{minipage}[h]{0.32\linewidth}
\center{\includegraphics[width=0.9\linewidth]{2.jpg}}
\end{minipage}
\hfill
\begin{minipage}[h]{0.32\linewidth}
\center{\includegraphics[width=0.9\linewidth]{3.jpg}}
\end{minipage}
\begin{minipage}[h]{1\linewidth}
\begin{tabular}{p{0.32\linewidth}p{0.32\linewidth}p{0.32\linewidth}}
\centering а) & \centering б) & \centering в) \\
\end{tabular}
\end{minipage}
\vspace*{-1cm}
\caption{Реализованное приложение}
\label{ris:correlationsignals}
\end{figure}

%%%%%%%%%%%%%%%%%%%%%%%%%%%%%%%%%
%%%                           %%%
%%% Заключение                %%%

\section*{Заключение}
\addcontentsline{toc}{section}{Заключение}

Так, нами был разработан функционал для генерации и сканирования QR-кодов - приложение, которое стало в настоящее время насущной необходимостью, поскольку с помощью него возможно считывание информации одним наведением цифровой видеокамеры без использования клавиатуры. 

Написание программы способствовало закреплению теоретического материала на практике. Приложение является логически завершенным. Также возможны изменения и добавления некоторых функциональных частей, которые можно реализовать в дальнейшем.

\end{document}
